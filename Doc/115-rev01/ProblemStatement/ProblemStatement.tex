\documentclass[11pt, oneside]{article}   	% use "amsart" instead of "article" for AMSLaTeX format
\usepackage{geometry}                		% See geometry.pdf to learn the layout options. There are lots.
\geometry{letterpaper}                   		% ... or a4paper or a5paper or ... 
%\geometry{landscape}                		% Activate for rotated page geometry
%\usepackage[parfill]{parskip}    		% Activate to begin paragraphs with an empty line rather than an indent
\usepackage{graphicx}				% Use pdf, png, jpg, or eps§ with pdflatex; use eps in DVI mode
								% TeX will automatically convert eps --> pdf in pdflatex		
\usepackage{float}
\usepackage{amssymb}
\usepackage{color}
\usepackage{hyperref}
\hypersetup{
    colorlinks,
    citecolor=black,
    filecolor=black,
    linkcolor=black,
    urlcolor=black
}

\usepackage{fancyhdr}
\usepackage{fancyhdr}
\fancyhead[L]{January 29, 2021}
\fancyhead[C]{SE 3XA3 Problem Statement}
\fancyhead[R]{AAA Solutions}
\pagestyle{fancy}

\usepackage{float}

%SetFonts

%SetFonts


\title{3XA3 L01 Group 15 Problem Statement
}
\author{Abraham Taha\\
		400170082
		\and
		Andrew Carvalino\\
		1311940
		\and
		Ali Tabar\\
		400183020}
\date{January 29th, 2021}							% Activate to display a given date or no date

%---------------------------------------------------------------------
\begin{document}
\maketitle
\newpage
%---------------------------------------------------------------------
\tableofcontents
\newpage
%---------------------------------------------------------------------

\section{Revision History}

\begin{table}[hp]
\caption{Revision History: Proof of Concept Plan}
\begin{center}
\label{tab:}
\begin{tabular}{|c|c|p{5cm}|c|}
\hline
\textbf{DATE} & \textbf{DEVELOPER} & \textbf{CHANGE} & \textbf{REVISION}\\
\hline
January 28, 2021 & Abraham Taha & Initial Draft & 0\\
\hline
January 28, 2021 & Andrew Carvalino & Initial Draft & 0\\
\hline
January 28, 2021 & Ali Tabar & Initial Draft & 0\\
\hline
\textcolor{red}{April 12, 2021} & \textcolor{red}{Abraham Taha} & \textcolor{red}{Updated section 2.1 "What problem are you trying to solve." also updated and improved section 2.3 "what is the context of the problem you are trying to solve.". All changes were made in red.} & \textcolor{red}{1}\\
\hline
\end{tabular}
\end{center}
\label{default}
\end{table}


\newpage
%---------------------------------------------------------------------
%\maketitle
\section{Problem Statement}
\subsection{What problem are you trying to solve?}

\newcommand*\apos{\textsc{\char13}}
We are working towards recreating this adaption of the old-school block game Tetris. \textcolor{red}{The problem that we are trying to solve is to create a software that can recognize a game grid with a predetermined shape flowing down. The software should also have the capability to detect when a full row of the grid has been filled with shapes and delete that row while increasing the users score. It should also have the functionality to take user inputs and move and or rotate the shapes based on those inputs.} As game developers, our goal is to add features that captivate users and enhance the product overall. We are changing this implementation to create the game on a web browser, with a cleaner and more modern user interface. Additionally, the original game is incomplete and can be improved upon with adding of more features, such as difficulty and other gameplay settings which can be configured by the user. One feature the original game has is to adjust the speed of the falling blocks - however, this is only achievable if the source code is edited. We will strive to fix this problem by adding buttons to the UI, increasing the user-friendliness.
\subsection{Why is this an important problem?}

The average user wanting to play the original game would not have the knowledge capable of executing the process to start the game, making it virtually unplayable. Most people would be annoyed with the fact that they’d have to locally install a few Python libraries, and then execute the command to run a .py file, all in a console. Our goal is to create a way where a person with zero background in programming can easily launch and play the game. Furthermore, the game as it currently exists is somewhat dull, and could use more gameplay features to lengthen user playtime and enjoyment. Though Tetris is a simple game with its mechanics, more variety can be added to potentially boost the user’s enjoyment, like a local multiplayer mode where two people can play against each other on the same device, or different difficulty settings, such as variations in the speed of the game or certain shapes which are dropped down into the screen.

\subsection{What is the context of the problem you are solving?}

We are looking to reproduce Tetris in a way that increases its accessibility to \textcolor{red}{users that have a PC capable of a Wi-Fi connection and can run a web browser. We are also focusing on ensuring maintainability, by implementing the code in such a way that it can be built upon easily, should future change occur. We will be using JavaScript, HTML and CSS to create the functionality and User Interface of the game. This will also allow our program to be used across any device that has web browser functionality.} The stakeholders in this reproduction will be the developers and the users. 


\end{document}  
