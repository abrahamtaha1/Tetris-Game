\documentclass[12pt]{article}

\usepackage{graphicx}
\usepackage{paralist}
\usepackage{amsfonts}
\usepackage{amsmath}
\usepackage{hhline}
\usepackage{booktabs}
\usepackage{multirow}
\usepackage{multicol}
\usepackage{tabularx}
\usepackage[normalem]{ulem}
\usepackage{xcolor}

\oddsidemargin 0mm
\evensidemargin 0mm
\textwidth 160mm
\textheight 200mm
\renewcommand\baselinestretch{1.0}

\pagestyle {plain}
\pagenumbering{arabic}

\newcounter{stepnum}

\usepackage{color}

\newif\ifcomments\commentstrue

\ifcomments
\newcommand{\authornote}[3]{\textcolor{#1}{[#3 ---#2]}}
\newcommand{\todo}[1]{\textcolor{red}{[TODO: #1]}}
\else
\newcommand{\authornote}[3]{}
\newcommand{\todo}[1]{}
\fi

\newcommand{\wss}[1]{\authornote{blue}{SS}{#1}}

\title{SE 3XA3: Module Interface Specification\\Tetrileet}

\author{Team 15, AAA Solutions
		\\ Student 1 Abdallah Taha, tahaa8
		\\ Student 2 Andrew Carvalino, carvalia
		\\ Student 3 Ali Tabar, sahraeia
}

\date{\today}

\begin{document}

\maketitle

This document is the Module Interface Specification (MIS) of AAA Solutions's Tetrileet.

\begin{table}[h!]
\caption{\bf Revision History}
\begin{tabularx}{\textwidth}{p{3cm}p{2cm}X}
\toprule {\bf Date} & {\bf Version} & {\bf Notes}\\
\midrule
March 17 2021 & 1.0 & Added all modules from Game Controller to End Game \\
March 18 2021 & 1.1 & Added all modules from BlockFall to Grid\\
\textcolor{red}{April 12 2021} & \textcolor{red}{1.2} & \textcolor{red}{Explained all of the state variables within the modules as well as explaining the exported types. Also added exceptions to rotate within the Game Controller Module. Furthermore, outputs in the access routines of the Game Controller were also updated. Block and Square were also explained throughout the document. All the changes were done in red to differentiate them.}\\
\bottomrule
\end{tabularx}
\end{table}

\newpage

\section* {Game Controller Module}

\subsection*{Module}

Game Controller Type

\subsection* {Uses}
Grid

\subsection* {Syntax}

\subsubsection* {Exported Constants}
N/A
\subsubsection* {Exported Types}
N/A

\subsubsection* {Exported Access Programs}

\begin{tabular}{| l | l | l | l |}
\hline
\textbf{Routine name} & \textbf{In} & \textbf{Out} & \textbf{Exceptions}\\
\hline
moveLeft & keyInput &  & Left\_Edge\_Grid \\
\hline
moveRight & keyInput  &  & Right\_Edge\_Grid \\
\hline
moveDown & keyInput &  & Taken\_Grid \\
\hline
rotate & keyInput &  &  \textcolor{red}{Right\_Edge\_Grid or Left\_Edge\_Grid or Taken\_Grid} \\
\hline
\end{tabular}

\subsection* {Semantics}

\subsubsection* {State Variables}
None 
\subsubsection* {Environment Variables}
keyInput: \{”key.W”, ”key.S”, ”key.A”, ”key.D”\}

\subsubsection* {State Invariant}
None
\subsubsection* {Assumptions}
\begin{itemize}
    \item Game Window is open and the start game button has been called.
\end{itemize}

\subsubsection* {Access Routine Semantics}

\noindent moveLeft(key.A):
\begin{itemize}
\item transition: $grid.moveLeft()$
\item output: None
\item exception: Left\_Edge\_Grid
\end{itemize}

\noindent moveRight(key.D):
\begin{itemize}
\item transition: $grid.moveRight()$
\item output: \textcolor{red}{None}
\item exception: Right\_Edge\_Grid
\end{itemize}

\noindent moveDown(key.S):
\begin{itemize}
\item transition: $grid.moveDown()$
\item output: \textcolor{red}{None}
\item exception: Taken\_Grid
\end{itemize}

\noindent rotate(key.W):
\begin{itemize}
\item transition: $grid.rotate()$
\item output: \textcolor{red}{None}
\item exception: \textcolor{red}{Right\_Edge\_Grid or Left\_Edge\_Grid or Taken\_Grid}
\end{itemize}

\subsection*{Local Constants}
None
\medskip
\newpage

\section* {Game Window Module}

\subsection*{Module}

Game Window Type

\subsection* {Uses}
None
\subsection* {Syntax}

\subsubsection* {Exported Constants}
N/A
\subsubsection* {Exported Types}
Block: \textcolor{red}{Block is a div in a HTML file which has not been tagged as a shape within the game}

\subsubsection* {Exported Access Programs}

\begin{tabular}{| l | l | l | l |}
\hline
\textbf{Routine name} & \textbf{In} & \textbf{Out} & \textbf{Exceptions}\\
\hline
draw &  &  &  \\
\hline
eraseBlock & & & \\
\hline
\end{tabular}

\subsection* {Semantics}

\subsubsection* {State Variables}

Square: \textcolor{red}{Square is a block div in HTML that has been recolored and tagged as a shape within the game}


\subsubsection* {Environment Variables}
None
\subsubsection* {State Invariant}
$|Square| == 200 $

\subsubsection* {Assumptions}

\begin{itemize}
    \item The HTML file is executed in a compatible browser.
\end{itemize}


\subsubsection* {Access Routine Semantics}

\noindent draw():
\begin{itemize}
\item transition: $Square \rightarrow Block$
\item output: None
\item exception: None
\end{itemize}

\noindent eraseBlock():
\begin{itemize}
\item transition: $Block \rightarrow Square$
\item output: None
\item exception: None
\end{itemize}

\newpage

\section* {Pause Game Module}

\subsection*{Module}

Pause Game Type

\subsection* {Uses}

Game Window\\
BlockFall

\subsection* {Syntax}

\subsubsection* {Exported Constants}
N/A
\subsubsection* {Exported Types}

$\mathbb{B}$ \textcolor{red}{This module exports a boolean to other modules telling them that the game is in the pause condition} 

\subsubsection* {Exported Access Programs}

\begin{tabular}{| l | l | l | l |}
\hline
\textbf{Routine name} & \textbf{In} & \textbf{Out} & \textbf{Exceptions}\\
\hline
Pause & keyInput &  &  \\
\hline
\end{tabular}

\subsection* {Semantics}

\subsubsection* {State Variables}
Paused: \textcolor{red}{Paused is a boolean that is true when the game pause condition is active and false otherwise} 

\subsubsection* {Environment Variables}

keyInput: \{”key.P”\}

\subsubsection* {State Invariant}

$Paused = True \lor False $

\subsubsection* {Assumptions}

N/A

\subsubsection* {Access Routine Semantics}

\noindent Pause(key.P):
\begin{itemize}
\item transition: $\textcolor{red}{(Paused \equiv True) \implies (Paused := False \lor Paused \equiv False)} \implies Paused := True$
\item output: None
\item exception: None
\end{itemize}

\newpage

\section* {Start Game Module}

\subsection*{Module}

Start Game Type

\subsection* {Uses}

Game Window, BlockFall

\subsection* {Syntax}

\subsubsection* {Exported Constants}
N/A
\subsubsection* {Exported Types}

$\mathbb{B}$ \textcolor{red}{This module exports a Boolean that tells the other modules that the game conditions have been reset and game has started}


\subsubsection* {Exported Access Programs}

\begin{tabular}{| l | l | l | l |}
\hline
\textbf{Routine name} & \textbf{In} & \textbf{Out} & \textbf{Exceptions}\\
\hline
Start & keyInput &  &  \\
\hline
\end{tabular}

\subsection* {Semantics}

\subsubsection* {State Variables}

Started: \textcolor{red}{Started is a boolean that is true when the game start button has been clicked and False otherwise} 


\subsubsection* {Environment Variables}
keyInput: \{”mouse.leftClick”\}

\subsubsection* {State Invariant}

$ Started = True \lor False $

\subsubsection* {Assumptions}

N/A

\subsubsection* {Access Routine Semantics}

\noindent Start(mouse.leftClick):
\begin{itemize}
\item transition: $ \textcolor{red}{(Started \equiv False)} \implies Started := True$
\item output: None
\item exception: None
\end{itemize}

\newpage

\section* {End Game Module}

\subsection*{Module}

End Game Type

\subsection* {Uses}

Game Window, Grid

\subsection* {Syntax}

\subsubsection* {Exported Constants}

N/A

\subsubsection* {Exported Types}

$\mathbb{B}$ \textcolor{red}{This module exports a boolean to tell other modules that the game has ended} 


\subsubsection* {Exported Access Programs}

\begin{tabular}{| l | l | l | l |}
\hline
\textbf{Routine name} & \textbf{In} & \textbf{Out} & \textbf{Exceptions}\\
\hline
gameOver &  &  & \\
\hline
\end{tabular}

\subsection* {Semantics}

\subsubsection* {State Variables}
Ended: \textcolor{red}{Ended is a boolean state variable within the module that is True when the game blocks haven't overflowed from the grid and False otherwise.} 

\subsubsection* {Environment Variables}
keyInput: \{”mouse.leftClick”\}
\subsubsection* {State Invariant}
$ Ended = True \lor False $
\subsubsection* {Assumptions}
Game has started and is in a running state.

\subsubsection* {Access Routine Semantics}

\noindent gameOver():
\begin{itemize}
\item transition: $ \textcolor{red}{(Ended \equiv False)} \rightarrow Ended := True$ 
\item output: None
\item exception: None
\end{itemize}

\newpage

\section* {BlockFall Module}

\subsection*{Module}
BlockFall Type
\subsection* {Uses}

Game Window, Grid, Shape

\subsection* {Syntax}

\subsubsection* {Exported Constants}
N/A
\subsubsection* {Exported Types}

N/A

\subsubsection* {Exported Access Programs}

\begin{tabular}{| l | l | l | l |}
\hline
\textbf{Routine name} & \textbf{In} & \textbf{Out} & \textbf{Exceptions}\\
\hline
falling &  &  & Taken\_Grid \\
\hline
\end{tabular}

\subsection* {Semantics}

\subsubsection* {State Variables}

None

\subsubsection* {Environment Variables}
None
\subsubsection* {State Invariant}

N/A

\subsubsection* {Assumptions}
Game has started and is in a running state, and is not paused.

\subsubsection* {Access Routine Semantics}

\noindent falling():
\begin{itemize}
\item transition: Game will call $grid.moveDown()$ every 1,000 milliseconds
\item output: none
\item exception: Taken\_Grid
\end{itemize}

\newpage

\section* {BlockStack Module}

\subsection*{Module}

BlockStack Type

\subsection* {Uses}

Shape, BlockFall

\subsection* {Syntax}

\subsubsection* {Exported Constants}
\textcolor{red}{None}
\subsubsection* {Exported Types}
N/A
\subsubsection* {Exported Access Programs}

\begin{tabular}{| l | l | l | l |}
\hline
\textbf{Routine name} & \textbf{In} & \textbf{Out} & \textbf{Exceptions}\\
\hline
freezeBlock &  &  & \\
\hline
\end{tabular}

\subsection* {Semantics}

\subsubsection* {State Variables}

N/A

\subsubsection* {Environment Variables}

N/A

\subsubsection* {State Invariant}

N/A

\subsubsection* {Assumptions}
Game has started and is in a running state, and is not paused.
\subsubsection* {Access Routine Semantics}

\noindent freezeBlock():
\begin{itemize}
\item transition: $Shape.block := $ "Taken"
\item output: None
\item exception: None
\end{itemize}

\newpage

\section* {Shape Module}

\subsection*{Module}

Shape Type

\subsection* {Uses}

Grid


\subsubsection*{Exported Constants}
N/A
\subsubsection*{Exported Types}

N/A

\subsubsection* {Exported Access Programs}

\begin{tabular}{| l | l | l | l |}
\hline
\textbf{Routine name} & \textbf{In} & \textbf{Out} & \textbf{Exceptions}\\
\hline
createShape &  &  & \\
\hline
\end{tabular}

\paragraph* {Semantics}

\subsubsection*{State Variables}

None

\subsubsection*{Environment Variables}

N/A

\subsubsection*{State Invariant}

N/A

\subsubsection*{Assumptions}

Game has started and is in a running state, and is not paused.

\subsubsection* {Access Routine Semantics}

\noindent createShape():
\begin{itemize}
\item transition: Takes coordinates from a predetermined array, and selects the square with the tag 'block', and gives them a colour based on the corresponding shape
\item output:
\item exception: None
\end{itemize}


\newpage

\section* {RowCheck Module}

\subsection*{Uses}

Grid

\subsection*{Syntax}

\subsubsection*{Exported Constants}
N/A
\subsubsection*{Exported Types}

N/A

\subsubsection* {Exported Access Programs}

\begin{tabular}{| l | l | l | l |}
\hline
\textbf{Routine name} & \textbf{In} & \textbf{Out} & \textbf{Exceptions}\\
\hline
rowCheck & & $\mathbb{B}$ & \\
\hline
rowClear & &  & \\
\hline
addScore & & $\mathbb{R}$ & \\
\hline
\end{tabular}

\paragraph* {Semantics}

\subsubsection*{State Variables}

clear: \textcolor{red}{Clear is a boolean that gets set to True after the blocks in a filled row get cleared and it is false otherwise.}\\ rowIndex: \textcolor{red}{This is a integer that corresponds to which index in the row the block is at.}\\ addToScore:\textcolor{red}{Integer value of 10 that gets added to total score when a clear gets set to True.}

\subsubsection*{Environment Variables}

None

\subsubsection*{State Invariant}

\begin{itemize}
    \item $Clear \equiv True \lor False$
    \item $0 \leq rowIndex \leq 20$
    \item $addToScore \equiv 10$
\end{itemize}

\subsubsection*{Assumptions}

N/A

\subsubsection* {Access Routine Semantics}

\noindent rowCheck():
\begin{itemize}
\item transition: $\forall~ rowIndex ~\in~ row ~|~ row[rowIndex].block ~\equiv~ 'taken' \implies clear := true$
\item output: clear
\item exception: None
\end{itemize}

\noindent RowClear():
\begin{itemize}
\item transition: $ \textcolor{red}{(clear := true)} \implies ~\forall~ rowIndex ~\in~ row ~|~ row[rowIndex].remove('block') ~\land~ \\row[rowIndex].remove('taken')$
\item output: None
\item exception: None
\end{itemize}

\noindent addScore():
\begin{itemize}
\item transition: $ \textcolor{red}{(clear == true)} \implies addToScore += 10$
\item output: addToScore
\item exception: None
\end{itemize}


\newpage

\section*{Grid}

\subsection*{Uses}

Game Window

\subsection*{Syntax}

\subsubsection*{Exported Constants}

Square

\subsubsection*{Exported Types}

N/A

\subsubsection* {Exported Access Programs}

\begin{tabular}{| l | l | l | l |}
\hline
\textbf{Routine name} & \textbf{In} & \textbf{Out} & \textbf{Exceptions}\\
\hline
createGrid() &  & &\\
\hline
displayShape() &  &  &\\
\hline
moveRight() &  &   &Right\_Edge\_Grid\\
\hline
moveLeft() &   &   &Left\_Edge\_Grid\\
\hline
moveDown() &  & & Taken\_Grid \\
\hline
rotate() & &  &Right\_Edge\_Grid $\land$ Left\_Edge\_Grid $\land$ Taken\_Grid \\
\hline
\end{tabular}

\paragraph* {Semantics}

\subsubsection*{State Variables}

[Square]:\textcolor{red}{This is an array of squares which are div in the HTML file these are what makeup the game grid}\\ width: \textcolor{red}{This is a constant which is the width of the game grid}\\ height: \textcolor{red}{This is a constant which is the height of the game grid}

\subsubsection*{Environment Variables}

N/A

\subsubsection*{State Invariant}

$width = 10 \land height = 20$\\
$|[Square]| = 200$

\subsubsection*{Assumptions}

N/A

\subsubsection* {Access Routine Semantics}

\noindent createGrid:
\begin{itemize}
\item transition: Creates a grid with a height of 20 squares and a width of 10 squares. By taking the squares from the Game Window and assigning them a 'Block' tag.
\item output: None
\item exception: None
\end{itemize}

\noindent displayShape():
\begin{itemize}
\item transition: Takes the predetermined square coordinates from an array that have the tag 'block' and recolors aforementioned squares. It then gives them a 'taken' tag.  
\item output: None
\item exception: None
\end{itemize}

\noindent moveRight():
\begin{itemize}
\item transition: Moves all squares with the tags 'block' and 'taken' from the current shape over by one column to the right. Then, recolours the newly taken squares and gives them a 'taken' tag. Proceeds to remove the colour and 'taken' tag from the squares that are no longer taken.
\item output: None
\item exception: None
\end{itemize}

\noindent moveLeft():
\begin{itemize}
\item transition: Moves all squares with the tags 'block' and 'taken' from the current shape over by one column to the left. Then, recolours the newly taken squares and gives them a 'taken' tag. Proceeds to remove the colour and 'taken' tag from the squares that are no longer taken.
\item output: None
\item exception: None
\end{itemize}

\noindent moveDown():
\begin{itemize}
\item transition: Moves all squares with the tags 'block' and 'taken' from the current shape over by one row down. Then, recolours the newly taken squares and gives them a 'taken' tag. Proceeds to remove the colour and 'taken' tag from the squares that are no longer taken.
\item output: None
\item exception: None
\end{itemize}

\noindent rotate():
\begin{itemize}
\item transition: Receives the change coordinates for the current shape from a predetermined array. Then, uses those coordinates to remove 'taken' tags and color from unused blocks after rotation. Finally, the new blocks are coloured and given a 'taken' tag. 
\item output: None
\item exception: None
\end{itemize}

\newpage


\end {document}
